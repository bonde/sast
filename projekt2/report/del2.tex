Vi har simuleret 5000 observationer fra den logaritmiske
normalfordeling med parametre $(5, 0.25)$. Plottet kan ses i figur \ref{log_plot_1}.
\begin{figure}[!h]
    \centering
    \includegraphics[width=0.8\textwidth]{../src/plot_1}
    \caption{Simulation af 5000 observationer fra den logaritmiske
    normalfordeling med parametre $(5, 0.25)$.}
    \label{log_plot_1}
\end{figure}

Ved brug af \textbf{R} har vi fundet stikprøvens gennemsnit, varians, spredning
og median. Disse ses i udskriften fra programmet i kodeboks \ref{r_output_1}.

\begin{lstlisting}[caption={Udskrift fra R-program},captionpos=b,label={r_output_1},numbers=left,
    frame=tb, breaklines=false, float=h]
[1] "Assignment 7"
[1] "Middelvaerdi"
[1] 167.1192
[1] "Varians"
[1] 7976.314
[1] "Standardafvigelse"
[1] 89.31021
[1] "Median"
[1] 147.5676
\end{lstlisting}

Ved at bruge formlerne fra Del 1, har vi regnet de teoretiske værdier for
gennemsnit, varians, spredning og median. De ses herunder.

\begin{align}
    E(S) & = e^{\mu+\sigma^{2}/2}\nonumber\\
         & = e^{5+0.125}\nonumber\\
         & = 168.1741
\end{align}
\begin{align}
    Var(S) & = (e^{\sigma^{2}} - 1)e^{2\mu+\sigma^2}\nonumber\\
    & = (e^{0.25} - 1)e^{10+0.25}\nonumber\\
    & = 8032.96
\end{align}
\begin{align}
    \sqrt{Var(S)} & = \sqrt{8032.96}\nonumber\\
    & = 89.62678
\end{align}
\begin{align}
    F^{-1}(0.5) & = e^{\mu}\nonumber\\
    & = e^{5}\nonumber\\
    & = 148.4132
\end{align}

Det ses, at de teoretiske værdier ligger meget tæt på dem som er blevet
udregnet fra stikprøven. Dette kan kan også ses ved at kigge på plottet i figur
\ref{log_plot_1}, hvor tæthedsfunktionen følger de observerede værdier.  Det
var også at forvente, når man har taget så stor en stikprøve.
