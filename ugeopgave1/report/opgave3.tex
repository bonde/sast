{

Metoden \texttt{assignment2\_3\_5()} printer ved kørsel gennemsnittet og
den emperiske varians for de simulerede kørsler ud. Dette gøres ved
metoder \texttt{mean()} og \texttt{var()} i \textbf{R}. Endvidere bruges
metoden \texttt{paste()} til at manipulere tekststrenge. Vi får følgende
udskrift:

\begin{verbatim}
[1] Mean for simulation with size 8 is: 2.029
[1] Var for simulation with size 8 is: 1.60376276276276
[1]
[1] Mean for simulation with size 20 is: 4.997
[1] Var for simulation with size 20 is: 3.55654754754755
[1]
[1] Mean for simulation with size 100 is: 25.087
[1] Var for simulation with size 100 is: 17.6550860860861
\end{verbatim}

Middelværdien for en binomialfordeling $\mathbf{X}$ med antalsparameter $n$ og
sandsynlighedsparamter $p$ er givet som $\textrm{E}(\mathbf{X}) = np$ og
variansen er givet ved $\textrm{Var}(\mathbf{X}) = np(1 - p)$. For
$\mathbf{X} \sim{ \textrm{bin} \left( n,p \right) }$, hvor $n = 8$ og $p
= 0.25$,
får vi da
\begin{eqnarray}
    \textrm{E}(\mathbf{X}) = np = 8\cdot0.25 = 2 \\
    \textrm{Var}(\mathbf{X}) = np(1 - p) = 8\cdot0.25(1 - 0.25) = 1.5
\end{eqnarray}
For $\mathbf{X} \sim{ \textrm{bin} \left( n,p \right) }$, hvor $n = 20$
og $p = 0.25$, får vi
\begin{eqnarray}
    \textrm{E}(\mathbf{X}) = 20\cdot0.25 = 5 \\
    \textrm{Var}(\mathbf{X}) = 20\cdot0.25(1 - 0.25) = 3.75
\end{eqnarray}
Endelig for $\mathbf{X} \sim{ \textrm{bin} \left( n,p \right) }$, hvor $n
= 100$ og $p = 0.25$, får vi
\begin{eqnarray}
    \textrm{E}(\mathbf{X}) = 100\cdot 0.25 = 25 \\
    \textrm{Var}(\mathbf{X}) = 100\cdot0.25(1 - 0.25) = 18.75
\end{eqnarray}

}
