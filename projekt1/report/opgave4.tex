{
\newcommand{\ud}{\, \mathrm{d}}
\newcommand{\mb}[1]{\mathbf{#1}}
\newcommand{\tr}[1]{\textrm{#1}}
\newcommand{\indep}{{\;\bot\!\!\!\!\!\!\bot\;}}

\newenvironment{nospace}%
{\noindent\ignorespaces}%
{\par\noindent%
  \ignorespacesafterend}
\subsection*{a)}
Vi betragter $k$ uafh\ae ngige stokastiske variable med antalsparameter $n$ og sandsynlighedsparameter $p$, dette vil sige
\[
\mb{X}_i \sim{ \tr{bin} \left( n,p \right) } , \tr{ for } i = 1,\dots,k
\]
Der skal nu findes middelv\ae rdien $\tr{E}(\overline{\mb{X}})$, hvor $\overline{\mb{X}} = \frac{1}{k} \sum_{i = 1}^k \mb{X}_i$, da $\mb{X}_i$ er binomialfordelt med antalsparameter $n$ og sandsynlighedsparameter $p$ er $\tr{E}(\mb{X}_i) = np$. Derfor f\o lger udregningen
\[
\tr{E}\mb{\overline{X}} = \tr{E} \left(\frac{1}{k}\sum_{i = 1}^k \mb{X}_i \right)
\]

\[
= \frac{1}{k}\sum_{i = 1}^k \tr{E}(\mb{X}_i)
\]

\[
= \frac{1}{k}k np
\]

\[
= np
\]

Vi skal nu finde det samme for variansen, dette g\o res p\aa f\o lgende m\aa de, vi ved at $\tr{Var}(\mb{\overline{X}}) = \tr{E}(\mb{\overline{X}^2}) - \tr{E}(\mb{\overline{X}})^2$ og der vides at $\tr{Var}(\mb{X}_i) = np(1-p)$.
\[
\tr{Var}(\mb{\overline{X}}) = \tr{Var}(\frac{1}{k} \sum_{i = 1}^k \mb{X}_i)
\]

\[
= \frac{1}{k^2} \sum_{i = 1}^k \tr{Var}(\mb{X}_i)
\]

\[
=\frac{1}{k^2} \sum_{i = 1}^k np(1-p)
\]
Her benyttes s\ae tning 3.7.13.
\[
= \frac{1}{k^2} k np(1-p)
\]
Derfor er

\[
\tr{Var}(\mb{\overline{X}}) = \frac{np(1-p)}{k}
\]

\subsection*{b)}
Den empiriske varians, kan findes ved at kigge p\aa\ opgave 3.26, og beskrives som
\[
s^2 = \frac{1}{n-1}\sum_{i = 1}^{n}\left(\mb{X}_i - \mb{\overline{X}}\right)^2
\]
Da vi betragter $k$ stokastiske variable fremfor $n$, har vi alts\aa\ at
\[
s^2 = \frac{1}{k-1}\sum_{i = 1}^{k}\left(\mb{X}_i - \mb{\overline{X}}\right)^2
\]
Vi skal nu finde middelv\ae rdien af dette, hvilket vil sige at vi skal kigge p\aa
\[
\tr{E}(s^2) = \tr{E}\left( \frac{1}{k-1}\sum_{i = 1}^{k}\mb{X}_i - \mb{\overline{X}}\right)^2
\]
\[
= \frac{1}{k-1}\sum_{i = 1}^{k} \tr{E}\left(\mb{X}_i - \mb{\overline{X}} \right)^{2}
\]
\[
= \frac{1}{k-1}\sum_{i = 1}^{k} \tr{E}\left(\mb{X}_{i}^{2} + \mb{\overline{X}}^{2} - 2\mb{X}_i\mb{\overline{X}} \right)
\]
Vi kan ved at bruge s\ae tning 3.7.6 og 3.7.4, dele ovenst\aa ende op og tage middel v\ae rdien af hvert enkelt led, s\aa ledes at vi har
\[
= \frac{1}{k-1} \sum_{i = 1}^{k} \left( \tr{E} \left( \mb{X}_{i}^{2} \right) + \tr{E} \left( \mb{\overline{X}}^{2} \right) - 2 \tr{E} \left( \mb{X}_i \mb{\overline{X}} \right) \right)
\]

Vi betragter nu kun det f\o rste led, alts\aa\ $E(\mb{X}_i^2)$. Til dette ved vi at der i henhold til 3.7.9 g\ae lder at
\[
\tr{Var}\left(\mb{X}\right) = \tr{E}\left( \mb{X}^2 \right) - \left( \tr{E} \left( \mb{X} \right) \right)^2
\]
Her isolerer vi $\tr{E}\left( \mb{X^2} \right) = \tr{Var}(\mb{X}) + \left( \tr{E} ( \mb{X} ) \right)^2$ fra f\o rste del af opgave 4 ved vi at $\tr{Var}\left( \mb{X}_i \right) = np(1-p)$ og $\left(\tr{E}\left(\mb{X}_i\right)\right)^2 = n^2p^2$
Derfor har vi at
\[
\tr{E}\left( \mb{X}_{i}^{2} \right) = np(1-p) + n^2p^2
\]
Det samme g\o r sig n\ae sten g\ae ldende for ledet $\tr{E}(\mb{\overline{X}}^{2} )$, her har vi fra f\o rste del af opgaven at
\[
\tr{Var}\left(\mb{\overline{X}} \right) = \frac{np(1-p)}{k} \tr{ og } \left(\tr{E}\left(\mb{\overline{X}}\right)\right)^2 = n^2p^2
\]
Derfor har vi at
\[
\tr{E}\left(\mb{\overline{X}}^{2} \right) = \frac{np(1-p)}{k} + n^2p^2
\]
Der betragtes nu det sidste led som har formen
\[
\tr{E}\left( \mb{X}_i \mb{\overline{X}}\right)
\]
\[
= \tr{E}\left( \mb{X}_i \frac{1}{k} \sum_{j = 1}^{k} \mb{X}_j \right)
\]
\[
= \frac{1}{k} \sum_{j = 1}^{k} \tr{E}\left( \mb{X}_i \mb{X}_j \right)
\]
Her kan vi i henhold til s\ae tning 3.7.7 g\o re f\o lgende, da $\mb{X}_i \indep \mb{X}_j$
\[
= \frac{1}{k} \left( \sum_{\substack{j = 1 \\ j \neq i}}^{k} \left( \tr{E} \mb{X}_i \tr{E} \mb{X}_j \right) + \tr{E} \left( \mb{X}_{i}^{2} \right) \right)
\]
Vi tager nu resultaterne fra f\o rste halv del af opgaven og bruger her.
\[
= \frac{1}{k}\left( \sum_{\substack{j = 1 \\ j \neq i}}^{k} \left( n^2p^2 + np \left( 1 - p \right) + n^2p^2 \right) \right)
\]
\[
= \frac{1}{k} \left( \left(k-1 \right) n^2p^2 + np \left( 1 - p \right) + n^2p^2 \right)
\]
Disse resultater inds\ae ttes nu i vores oprindelige formel
\[
\frac{1}{k-1}\sum_{i=1}^{k}\left( \tr{E}\left(\mb{X}_i^2\right) + \tr{E}\left( \mb{\overline{X}}^2\right) - 2\tr{E}\left(\mb{X}_i\mb{\overline{X}} \right) \right)
\]
\[
= \frac{1}{k-1}\sum_{i=1}^{k} \left( np(1-p)+n^2p^2 + \frac{np(1-p)}{k} + n^2p^2 - 2\frac{1}{k} \left( \left( k-1 \right) n^2p^2 + np \left( 1-p \right) + n^2p^2 \right) \right)
\]
\[
= \frac{1}{k-1} \left( \left( knp \left( 1-p \right) + kn^2p^2 + np(1-p) + kn^2p^2 \right) - 2 \left( \left( k-1 \right) n^2p^2 + np \left( 1-p \right) + n^2p^2\right) \right)
\]
\[
= \frac{1}{k-1} \left( knp \left( 1-p \right) + kn^2p^2  + np \left( 1-p \right) + kn^2p^2 - 2 kn^2p^2 +2n^2p^2 - 2np \left( 1-p \right)-2n^2p^2 \right)
\]

\[
= \frac{1}{k-1} \left( knp \left( 1-p \right) - np \left( 1-p \right) \right)
\]

\[
= \frac{1}{k-1} \left( \left( k-1 \right) np \left( 1-p \right) \right)
\]

\[
= np \left( 1-p \right)
\]
}
